\chapter{Parties prenantes du projet}

Ce chapitre va présenter tous les stakeholders, de près ou de loin, du projet. Nous avons décidé d'en faire un chapitre à part entière pour une question de clarté. Toutes les parties prenantes vont être explicitées dans les sections ci-dessous.

\section{Partenaires}

Dans les parties prenantes de notre système, on retrouve également les grosses chaines comme Basic Fit avec qui on peut établir un partenariat pour financer et promouvoir le projet. Nous pouvons également établir un partenariat avec des marques de sport connus. Dans les sponsors, on retrouvera aussi  des investisseurs qui pourront recevoir une part des bénéfices du projet. Les coachs sportifs appartenant aux firmes partenaires participent également à la réalisation de l'application puisque leur expertise sera nécessaire pour le développement de celle-ci afin qu'il y ait une bonne connaissance du domaine sportif.

\section{Utilisateurs du produit}

Les clients seront des personnes du public sportif ou des personnes voulant simplement se remettre en forme. 

L'application est destinée à toutes personnes voulant débuter dans un sport sans avoir nécessairement les connaissances. Principalement, on retrouvera des clients de grosse chaine de salle de sport comme \textit{Basic Fit} ou encore \textit{Jims Fitness}. L'application devrait moins convenir aux personnes ayant déjà une certaine expertise comme les personnes pratiquant le sport depuis un certain temps. De même, l'application ne couvre pas seulement les sports en salle de fitness. Elle cherche à couvrir un maximum de sports individuels.

\section{Stakeholders négatifs}

Il existe également des parties prenantes négatives qui seront contre un tel projet. Bien que certains coach sportifs, faisant partie des partenaires, devraient partager leur expertise du domaine, beaucoup iront à l'encontre du projet. Effectivement, les experts sondés appartenant à une chaîne indépendante de taille clairement plus petite que nos partenaires ne sont pas convaincus qu'un tel projet puisse se montrer concluant. 

Et d'autre part les coachs/gestionnaires de salle qui sont contre l'idée de créer un système qui pourrait les remplacer, ou contre l'utilisation des technologies en argumentant que la dimension sociale risque de disparaître. 

\section{Concurrents}

Les principaux concurrents sont d'une part les applications et sites existants à ce sujet. 

\section{Avantage concurrentiel}

Tout d'abord, on utilise des notions de \textit{machine learning} pour l'utilisateur afin de raffiner les séances proposées. En conséquent,  nous nécessitons une grande quantité d'échantillons de données. Pour le lancement de notre application, le mieux est de réaliser des formulaires en ligne que les clients des entreprises partenaires comme \textit{Basic Fit} ou \textit{Jims Fitness} pourraient remplir et on récupère ces données comme échantillons. Dès que le produit est lancé, on commence à récupérer les données réelles soumises par les utilisateurs afin de remplacer petit à petit les échantillons de données. On essaie alors de faire un bon compromis entre ces échantillons et les données des utilisateurs pour améliorer ses séances tout en se focalisant au fur et à mesure sur les données "réelles".

Tout type de sport individuel est visé, donc on remarque qu'il n'existe actuellement pas de réelle concurrence présentant les mêmes services. Les sites et applications existants dans ce domaine sont assez incomplets, statiques, et ne tiennent pas spécialement compte des performances précédentes. 

Un des autres avantages de notre produit, est l'utilisation de la \textit{gamification} dans le but de motiver l'utilisateur à poursuivre ses efforts. En effet, son but est de rendre les actions à effectuer plus ludiques, cela favoriserait également l'utilisateur à se prendre au jeu et de mener à une consultation plus fréquente, ce qui impliquerait un meilleur aboutissement de ses divers objectifs. Les mécaniques de \textit{gamification} se résument à des badges et des trophées soulignant le nombre de séances finies d'affilée, au total depuis l'inscription, les jours de connexion consécutifs,... Ces badges et trophées donneront des points, points qui seront liés au profil public de l'utilisateur.