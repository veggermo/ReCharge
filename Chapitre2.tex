\chapter{Parties prenantes du projet}

Ce chapitre va présenter tous les stakeholders, de près ou de loin, du projet. Nous avons décidé d'en faire un chapitre à part entière pour une question de clarté. Développer stakeholders

\section{Client, customer and other stakeholders}

Les clients seront des personnes du public sportif ou des personnes voulant simplement se remettre en forme. 

Dans les parties prenantes de notre système, on retrouve également les grosses chaines comme Basic Fit avec qui on peut établir un partenariat pour financer et promouvoir le projet. Nous pouvons également établir un partenariat avec des marques de sport connus. Dans les sponsors, on retrouvera aussi  des investisseurs qui pourront recevoir une part des bénéfices du projet. Les coachs sportifs participent également à la réalisation de l'application puisque leur expertise sera nécessaire pour que l'application ait une bonne connaissance du domaine. 

Il existe également des parties prenantes négatives qui seront contre un tel projet. Bien que certains coach sportifs, faisant partie des partenaires, devraient partager leur expertise du domaine, beaucoup iront à l'encontre du projet. Effectivement, les experts sondés appartenant à une chaîne indépendante de taille clairement plus petite que nos partenaires ne sont pas convaincus qu'un tel projet puisse se montrer concluant. 

\section{Utilisateurs du produit}

L'application est destinée à toutes personnes voulant débuter dans un sport sans avoir nécessairement les connaissances. Principalement, on retrouvera des clients de grosse chaine de salle de sport comme \textit{Basic Fit} ou encore \textit{Jims Fitness}. L'application devrait moins convenir aux personnes ayant déjà une certaine expertise comme les personnes pratiquant le sport depuis un certain temps. De même, l'application ne couvre pas seulement les sports en salle de fitness. Elle cherche à couvrir un maximum de sports individuels.

\section{Concurrents}

Les principaux concurrents sont d'une part les applications et sites existants à ce sujet. Et d'autre part les coachs/gestionnaires de salle qui sont contre l'idée de créer un système qui pourrait les remplacer, ou contre l'utilisation des technologies en argumentant que la dimension sociale risque de disparaître. 

\section{Avantage concurrentiel}

Tout d'abord, on utilise des notions de \textit{machine learning} pour l'utilisateur afin de raffiner les séances proposées.On vise tout type de sport individuel donc il n'existe pas de réelle concurrence présentant les mêmes services. Les sites et applications existants dans ce domaine sont assez incomplets, statiques, et ne tiennent pas spécialement compte des performances précédentes. On a besoin d'une grande quantité d'échantillons de données. 

Pour le lancement de notre application le mieux c'est de réaliser des formulaires en ligne que les clients des grosses firmes comme Basic Fit pourrait remplir et on les récupérer comme échantillons. Une fois l'application lancée, on essaierait de récupérer les données des utilisateurs pour remplacer petit à petit les échantillons de départ. On essaie alors de faire un bon compromis entre ces échantillons et les données d'un utilisateur pour améliorer ses séances.

Utilisation de la gamification (badges, trophées) pour motiver l'utilisateur à continuer ses efforts.