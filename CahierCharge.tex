\documentclass{report}



\usepackage[latin1]{inputenc} % un package

\usepackage[T1]{fontenc}      % un second package

\usepackage[francais]{babel}  % un troisi�me package

\begin{document}

\chapter{Project driver}

\section{The purpose of the project}

\section{Client, customer and other stakeholders}

L'application est destin� � toute personne voulant d�buter dans un sport sans avoir n�cessairement les connaissances. Principalement, on retrouvera des clients de grosse chaine de salle de sport comme \textit{Basic Fit} ou encore \textit{Jims Fitness}. L'application devrait moins convenir aux personnes ayant d�j� une certaine expertise comme les personnes pratiquant le sport depuis un certain temps. De m�me, l'application ne couvre pas seulement les sports en salle de fitness. Elle cherche � couvrir un maximum de sports individuels.

\section{Users of the product}

\chapter{Project constraints}

\section{Mandated constraints}

\section{Naming conventions and definitions}

\section{Relevant facts and assumptions}

\chapter{Functional requirements}

\section{The scope of the work}

\section{The scope of the product}

\section{Functional and data requirements}

\chapter{Non-functional requirements}

\section{Look and feel requirements}

\section{Usability and humanity requirements}

\section{Performance requirements}

\section{Operational requirements}

\section{Maintainability and support requirements}

\section{Security requirements}

\section{Cultural and political requirements}

\section{Legal requirements}     

\chapter{Project issues}

\section{Open issues}

\section{Off-the-shelf solution}

\section{New problems}

\section{Tasks}

\section{Cutover}

\section{Risks}

\section{Costs}

\section{User documentation and training}

\section{Waiting room}

\section{Ideas for solutions}

\end{document}