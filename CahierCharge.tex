\documentclass{report}



\usepackage[latin1]{inputenc} 

\usepackage[T1]{fontenc}      

\usepackage[francais]{babel}  

\begin{document}

\chapter{Project driver}

\section{The purpose of the project}

Le projet consiste � fournir au client un planificateur de s�ance de sport. Plus concr�tement, le planificateur serait une application mobile qui permettrait � son utilisateur d'avoir un suivi sur ses s�ances de sport et sa progression ainsi que d'avoir ses s�ances planifi�es automatiquement sans qu'il ait besoins des connaissances n�cessaires � la bonne pratique du sport. 

\section{Client, customer and other stakeholders}

Les clients seront des personnes du public sportif

Dans les parties prenantes de notre syst�me, on retrouve �galement les grosses chaines comme Basic Fit avec qui on peut �tablir un partenariat pour financer et promouvoir le projet. Nous pouvons �galement �tablir un partenariat avec des marques de sport connus. Dans les sponsors, on retrouvera aussi  des investisseurs qui pourront recevoir une part des b�n�fices du projet. Les coachs sportif participent �galement � la r�alisation de l'application puisque leur expertise sera n�cessaire pour que l'application ai une bonne connaissance du domaine. 

Il existe �galement des parties prenantes n�gatives qui seront contre un tel projet. Bien que certains coach sportifs devraient partager leur expertise du domaine, beaucoup iront � l'encontre du projet puisqu'il ne sont pas convaincu qu'un tel projet puisse se montrer concluant. De m�me, les petites salles de sport local sont �galement contre ce projet pour les m�me raisons. 

\section{Users of the product}

L'application est destin� � toute personne voulant d�buter dans un sport sans avoir n�cessairement les connaissances. Principalement, on retrouvera des clients de grosse chaine de salle de sport comme \textit{Basic Fit} ou encore \textit{Jims Fitness}. L'application devrait moins convenir aux personnes ayant d�j� une certaine expertise comme les personnes pratiquant le sport depuis un certain temps. De m�me, l'application ne couvre pas seulement les sports en salle de fitness. Elle cherche � couvrir un maximum de sports individuels.

\chapter{Project constraints}

\section{Mandated constraints}

\section{Naming conventions and definitions}

\section{Relevant facts and assumptions}

\chapter{Functional requirements}

\section{The scope of the work}

\section{The scope of the product}

\section{Functional and data requirements}

L'utilisateur de l'application doit pouvoir g�n�rer une s�ance de sport qui convient � ses objectifs

\chapter{Non-functional requirements}

\section{Look and feel requirements}

\section{Usability and humanity requirements}

\section{Performance requirements}

\section{Operational requirements}

\section{Maintainability and support requirements}

\section{Security requirements}

\section{Cultural and political requirements}

\section{Legal requirements}     

\chapter{Project issues}

\section{Open issues}

\section{Off-the-shelf solution}

\section{New problems}

\section{Tasks}

\section{Cutover}

\section{Risks}

\section{Costs}



\section{User documentation and training}

\section{Waiting room}

\section{Ideas for solutions}

\end{document}