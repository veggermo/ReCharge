\chapter{Objectifs du projet}

Ce chapitre va principalement servir à introduire le but du projet et le plan business prévu pour le développement de ce produit.

\section{But du projet}

Le projet consiste à fournir au client un planificateur de séances de sport. Plus concrètement, le planificateur se décrit comme une application mobile qui permettrait à l'utilisateur d'organiser ses séances d'entraînement de la manière la plus optimale possible, sans que l'utilisateur n'ait nécessairement les connaissances d'un expert dans le domaine. La séance sera tout d'abord générée selon les caractéristiques de l'utilisateur (poids, taille, âge, morphologie). Ensuite, lorsque cette séance est finie, les performances et les objectifs sont mis à jour. Et les séances futures s'organiseront en fonction des performances préalables de l'utilisateur. 

L'application vise les personnes sportives ou voulant faire du sport, plus particulièrement les personnes qui souhaite se lancer dans un sport sans nécessairement savoir comment débuter. L'application peut moins convenir pour un expert pratiquant son sport depuis plusieurs années. Elle se limite aux sports individuels tels que la natation, la course à pied, la musculation ou encore le cyclisme. 

\section{Plan business du projet}

Au niveau business, l'idée principale est de faire un partenariat avec une franchise réputée comme \textit{Basic Fit}, par exemple. Cette compagnie ferait office d'investisseur afin de pouvoir lancer le produit sur le marché. En contrepartie, des publicités à leur effigie seront présentes, tout en gardant une navigation agréable sur l'application. Par exemple, le service renseignera les salles de sports partenaires les plus proches avec l'affluence et la disponibilité des machines. Dès que le produit sera devenu rentable, les revenus entraînés par celui-ci seront partagés avec les investisseurs afin que ceux-ci aient un retour sur l'investissement. Les partenaires feront également en sorte de faire la publicité de l'application dans leurs locaux, leurs réseaux sociaux ou même lors d'inscription d'un nouveau membre.

L'application de base serait disponible gratuitement sur les plateformes de téléchargement d'applications mobiles (Play Store, App Store). Certains contenus déblocables seront eux payants car ils nécessiteront de l'équipement nécessaire, comme une montre connectée afin de calculer les pulsations cardiaques, à lier avec l'application.