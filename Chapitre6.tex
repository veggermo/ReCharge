
\chapter{Problèmes du projet}

\section{Questions sans réponses}

Un des premiers souci est qu'on ne sait pas si nous disposerons de suffisamment d'échantillons pour réaliser le(s) modèle(s). Il faut se demander s'il faut si nous devons récolter nous même des échantillons ou si nous disposons de suffisamment de données sur des base de données disponibles.

D'autre part, on ne connait pas la demande du marché pour ce genre de service. On ne sait pas si la demande sera assez important pour que l'application gagne suffisamment en succès.

De même, on n'est pas sûr de pouvoir recruter suffisamment d'experts sportifs, qui seront majoritairement des coach. En effet, ce projet peut être vu comme un frein à leur activité professionelle. 

Enfin, on ne sait pas si les coût de création et de support seront assez rentable.  

\section{Solution existante}

Concernant la récupération des données pour le(s) modèle(s), il existe des site de base de données destinés aux machine learning comme textit{https://mldb.ai/} qui peuvent nous permettre de réaliser le(s) modèle(s) de machine learning.

\section{Nouveaux problèmes}

\section{Tâches à faire}

Plusieurs étapes rassemblant certaines tâches sont à prendre en compte. 

\subsection*{Partenariat et recrutement}

Il faut, avant de pouvoir commencer le développement, chercher plus concrètement quelles entreprises pourraient être intéressés. Basic-Fit est une bonne entreprise mais il peut exister d'autres entreprise intéressés. Il faudra par la suite contacter ses firmes pour proposer et négocier un partenariat avec.
Une fois les partenariats établis, il faudra recruter les experts sportifs qui vont aider à les développeur à remplir au mieux les exigences fonctionnels de l'application. Ces experts peuvent venir de nos partenaires mais il s'agira principalement de coach. 

\subsection*{Conception de l'application et établissement du/des modèle(s)}

L'application doit être modélisé et les modules nécessaires doivent être identifié pour chaque fonctionnalité de l'application. Dans cette étape de conception le(s) modèle(s) et les algorithmes de machine learning doivent être établie. 

\subsection*{Implémentation de l'application}

Une fois les modules de l'application identifié et conceptualisé, il ne reste plus qu'à l'implémenter. 
  
\section{Cutover}

\section{Risks}

Plusieurs risques sont à prendre en compte. Tout d'abord, les bases de données peuvent être piratés à des buts commerciaux (récupération de données pour de la publicité privé). 

D'autre part, on risque d'avoir un manque d'experts sportifs puisque beaucoup pourront refuser notre offre de recrutement du fait que notre projet est un frein à leurs activités.

Un dernier risque est que les coûts de création et de support ne sont pas assez rentable.


\section{Costs}

L'élaboration de \textit{SportEasy} fait intervenir divers coûts :

\begin{itemize}

\item Les coûts de conception, d'implémentation et de déploiement de l'application

\item Les coûts des services des experts qui vont nous aider à développer et optimiser les performances de l'application en partageant leur expertise.

\item Les coûts de maintenance de l'application : Ils seront à assurer à partir du premier \textit{release} de l'application. La maintenance se fera en fonction du feedback reçu par les utilisateurs afin d'avoir un produit flexible et qui essaie de répondre le mieux possible aux attentes des utilisateurs.

\item Les coûts de \textit{hosting} pour tout ce qui concerne les serveurs et les éventuels data centers.

\end{itemize}

\section{User documentation and training}

\section{Waiting room}

\section{Ideas for solutions}

%L'application va nécessiter une grande quantité de données pour pouvoir réaliser les algorithmes. Pour trouver des données pertinentes, on pourrait réaliser un modèle basé sur de données de plusieurs sportif, chaque jeu de données d'un sportif étant classé selon sa morphologie, sa fréquence d'activité sportive et le nombre d'années passés à pratiquer son sport. Il serait pertinent de réaliser des échantillons sur base d'enquête réaliser auprès de plusieurs sportif à l'aide de nos partenaires (par exemple, si nous arrivons à avoir Basic-Fit dans nos partenaires il peut être intéressant de réaliser des enquêtes auprès de leurs clients).