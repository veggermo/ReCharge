
\chapter{Problèmes du projet}

\section{Questions sans réponses}

Un des premiers soucis est qu'on ne sait pas si nous disposerons de suffisamment d'échantillons pour réaliser le(s) modèle(s). Il faut se demander s'il faut si nous devons récolter nous même des échantillons ou si nous disposons de suffisamment de données sur des base de données disponibles.\\

D'autre part, on ne connait pas la demande du marché pour ce genre de service. On ne sait pas si la demande sera assez important pour que l'application gagne suffisamment en succès.\\

De même, on n'est pas sûr de pouvoir recruter suffisamment d'experts sportifs, qui seront majoritairement des coach. En effet, ce projet peut être vu comme un frein à leur activité professionnelle.\\ 

Enfin, on ne sait pas si les coût de création et de support seront supportables.  

\section{Solution existante}

Concernant la récupération des données pour le(s) modèle(s), il existe des site de base de données destinés aux \textit{machine learning} comme textit{https://mldb.ai/} qui peuvent nous permettre de réaliser le(s) modèle(s) de \textit{machine learning}.

\section{Nouveaux problèmes}

\section{Tâches à faire}

Plusieurs étapes rassemblant certaines tâches sont à prendre en compte. 

\subsection*{Partenariat et recrutement}

Il faut, avant de pouvoir commencer le développement, chercher plus concrètement quelles entreprises pourraient être intéressées. Basic-Fit est une bonne entreprise mais il peut exister d'autres entreprises intéressés. Il faudra par la suite contacter ses firmes pour proposer et négocier un partenariat avec.\\
Une fois les partenariats établis, il faudra recruter les experts sportifs qui vont aider à les développeur à remplir au mieux les exigences fonctionnels de l'application. Ces experts peuvent venir de nos partenaires mais il s'agira principalement de coach. 

\subsection*{Conception de l'application et établissement du/des modèle(s)}

L'application doit être modélisée et les modules nécessaires doivent être identifiés pour chaque fonctionnalité de l'application. Dans cette étape de conception, le(s) modèle(s) et les algorithmes de \textit{machine learning} doivent être établies. 

\subsection*{Implémentation de l'application}

Une fois les modules de l'application identifiés et conceptualisés, il ne reste plus qu'à implémenter l'application. 

\section{Contrôle final de qualité}

Afin de contrôler la qualité de notre application, nous la testerons auprès de nos partenaires. Si nos partenaires comptent des salles de fitness low-cost comme Basic-Fit ou Jims Fitness, ils pourront directement la faire tester auprès de leur clients. Ces clients testeront l'application jusqu'à une échéance déterminée. Une fois cette échéance terminée, ils devront donner une appréciation et un \textit{feedback} sur l'application.\\

Cette version \"beta\" pourra être corrigée s'il y a des retours négatifs sur la qualité du produit. Toutefois, il s'agira juste de corrections de bugs et d'améliorations d'éléments non-fonctionnels. Aucun élément fonctionnel ne sera modifié. 

\section{Risques}

Plusieurs risques sont à prendre en compte : \\

Tout d'abord, les bases de données peuvent être piratées à des fins commerciales (Ex : récupération de données pour de la publicité privée).\\

D'autre part, on risque d'avoir un manque d'experts sportifs puisque beaucoup pourront refuser notre offre de recrutement du fait que notre projet est un frein à leurs activités.\\

Un dernier risque est que les coûts de création et de support ne sont pas assez rentable.

\section{Coûts}

L'élaboration de \textit{SportEasy} fait intervenir divers coûts :

\begin{itemize}

\item Les coûts de conception, d'implémentation et de déploiement de l'application

\item Les coûts des services des experts qui vont nous aider à développer et optimiser les performances de l'application en partageant leur expertise.

\item Les coûts de maintenance de l'application : Ils seront à assurer à partir du premier \textit{release} de l'application. La maintenance se fera en fonction du feedback reçu par les utilisateurs afin d'avoir un produit flexible et qui essaie de répondre le mieux possible aux attentes des utilisateurs.

\item Les coûts de \textit{hosting} pour tout ce qui concerne les serveurs et les éventuels data centers.

\end{itemize}

\section{Documentation et formation pour l'utilisateur}

L'application ne nécessite aucune formation. Un didacticiel est réalisé à sa première utilisation. À sa première utilisation, l'utilisateur doit disposer d'un compte. Il peut soit s'inscrire soit utiliser sont compte Google ou Facebook pour se connecter.\\

L'utilisateur devra ensuite remplir un formulaire où il renseignera ses caractéristiques physiques (poids, taille, ...) ainsi que son objectif. L'honnêteté de l'utilisateur est primordial car la création de la séance va dépendre de ces données. Une fois l'inscription et l'entrée des caractéristiques physiques terminées, l'utilisateur peut utiliser l'application pour réaliser des séances de sport.\\

Chaque fois que l'utilisateur veut avoir recours à cette fonctionnalité, il devra entrer plusieurs paramètres, à savoir l'intensité de son entrainement, la durée et le sport choisi pour son entrainement. L'application lui génère alors une liste d'exercice que l'utilisateur peut  consulter à tous moment durant sa séance pour savoir en permanence quels exercices il doit faire.\\

Durant sa séance, l'utilisateur disposera d'une icône "renseignement" à côté de chaque exercice sur laquelle il pourra cliquer s'il souhaite de plus amples informations concernant l'exercice à réaliser.\\

De même, s'il souhaite supprimer un exercice, il dispose d'une icône "Supprimer" qu'il pourra utiliser pour ensuite sélectionner le(s) exercice(s) qu'il ne souhaite pas réaliser.\\

Pour finir, une fois son entrainement terminé, l'utilisateur valide sa séance et un formulaire lui est alors proposé. L'utilisateur donne alors une appréciation de sa séance à travers ce formulaire. La prochaine séance sera alors calculée en fonction des résultats des séances précédentes.

\section{Waiting room}

\section{Ideas for solutions}
